\documentclass[a4paper]{article}
%% Language and font encodings
\usepackage[english]{babel}
\usepackage[utf8x]{inputenc}
\usepackage[T1]{fontenc}

%% Sets page size and margins
\usepackage[a4paper]{geometry}

%% Useful packages
\usepackage{enumitem}
\usepackage{amsmath}
\usepackage{amsfonts}
\usepackage{listings}
\usepackage{graphicx}
\usepackage[colorinlistoftodos]{todonotes}
\usepackage[colorlinks=true, allcolors=blue]{hyperref}
\newenvironment{proof}{\paragraph{Proof:}}{\hfill$\square$}
\title{PVS Assignment 1}
\author{Siddharth G}


\date{\vspace{-5ex}}

\begin{document}
\maketitle

\itemize

\item \textbf{Q3}
\begin{verbatim}
data BAG:= P Nat Nat  % The Prouct type. The first coordinate indicates the number of 
                      % black balls, and the second indicates the number of whit balls.

f::BAG -> BAG
f (P m n)
	| m+n<=1 = P m n
	| otherwise = case () of 
				  () | a<m && b<m = f(P (m-2) (n+1))                 % both black
				  	  | (a<m && b>=m) || (a>=m && b<m) = f(P m (n-1)) % different colours
				  	  | otherwise = f(P m (n-1))                      % both white

       where (a,b) = randomPair(m+n)                 % <m is black, else white
\end{verbatim}

What is the colour of the last ball? - 
\begin{verbatim} first(f (P m n)) \end{verbatim} 
1 indicates a black ball and 0 indicates a white ball.

Inductive proof:
\begin{proof}
Let us induct on $m+n$. 
Base cases: $m+n=0$ or $1$. Trivial since  \begin{verbatim} f (P m n) = P m n \end{verbatim} and there are no draws from the bag.

Induction step: Assume the hypothesis for $m+n=k$. \\
 Let $m+n=k+1$, 
 $randomPair$ chooses two (distinct) balls. The $case ()$ condition appropriately adds an extra ball whose colour depends on $a$ and $b$. The new arguments $m'$ and $n'$  have the property that $m'+n'=k$ in all the three cases of $case()$. Thus $f$ satisfies the specification by the induction hypothesis.
\end{proof}
\newpage

\item \textbf{Q4}
\begin{verbatim}
	findMajority :: [Char] -> Char
	findMajority L = aux L 'F' 0
	
	aux :: [Char] -> Char -> Int -> Char
	aux [] x i = x
	aux v:vs x i
		| i=0 = aux vs v 1
		| x == v = aux vs v (i+1)
		| otherwise = aux vs x (i-1)
\end{verbatim} 

\begin{proof}
	Induct on size of the list of votes (L).
	Base case: $|L|=1$. 
	\begin{verbatim}
	findmajority [x] = aux L 'F' 0 = x  
	\end{verbatim} 
	which is the majority.	
	\\ \\
	Induction step: Assume correctness for $|L|=k$. \\
	Let $|L|=k+1$. Assume that the majority element for the first $k$ elements is $x$. Then, $i>0$ and the number of elements $ \neq x$ are at most $k/2-i$, and regardless of the last vote, the majority remains with $x$ (since we are guaranteed a majority). \\
	If $i=0$, then $x$ is not the majority element. Since we are guaranteed a majority at the end of the list, and we have no majority for the first $k$ elements, the majority is induced by the last element of $L$, which is exactly what $aux$ assigns as the majority element (first guarded line).  
	\end{proof}
	
\end{document}