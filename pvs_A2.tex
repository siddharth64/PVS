\documentclass[a4paper]{article}
%% Language and font encodings
\usepackage[english]{babel}
\usepackage[utf8x]{inputenc}
\usepackage[T1]{fontenc}

%% Sets page size and margins
\usepackage[a4paper]{geometry}

%% Useful packages
\usepackage{enumitem}
\usepackage{amsmath}
\usepackage{amsfonts}
\usepackage{listings}
\usepackage{graphicx}
\usepackage[colorinlistoftodos]{todonotes}
\usepackage[colorlinks=true, allcolors=blue]{hyperref}
\newenvironment{proof}{\paragraph{Proof:}}{\hfill$\square$}
\title{PVS Assignment 2 (liftif)}
\author{Siddharth G}


\date{\vspace{-5ex}}

\begin{document}
\maketitle

\itemize

\item \emph{1(a):} \textbf{assert} work on \textbf{if}-expressions only if the \textbf{if} condition can be simplified to \textit{true} or simplified to \textit{false} (prover-guide pg. 80). Since this is not possible in this case, \textbf{assert} does not work.

\item \emph{1(b):} pg. 45 of the prover-guide states that \textbf{split} works on \textbf{IF(B,C,D)} in the consequent by collecting formulas of the sort $ B \supset C$ or $\neg B \supset D$. Since neither of these occur in \textit{iflemma1}, \textbf{split} does not have an effect. \textbf{prop} repeatedly uses \textbf{split} and \textbf{flatten}, which for similar reasons, has no effect. \textbf{ground} uses \textbf{prop} and \textbf{assert}, and since neither of them work, neither does \textbf{ground.}

\item \emph{1(d):} \textbf{lift-if *} pulls the equality into the \textbf{IF} structure, by imposing the equality in each possibility of the \textbf{IF}-expressions instead of imposing the equality of the two outermost \textbf{IF}-expressions themselves. 
	
\end{document}